% Options for packages loaded elsewhere
% Options for packages loaded elsewhere
\PassOptionsToPackage{unicode}{hyperref}
\PassOptionsToPackage{hyphens}{url}
\PassOptionsToPackage{dvipsnames,svgnames,x11names}{xcolor}
%
\documentclass[
  letterpaper,
  DIV=11,
  numbers=noendperiod]{scrartcl}
\usepackage{xcolor}
\usepackage{amsmath,amssymb}
\setcounter{secnumdepth}{-\maxdimen} % remove section numbering
\usepackage{iftex}
\ifPDFTeX
  \usepackage[T1]{fontenc}
  \usepackage[utf8]{inputenc}
  \usepackage{textcomp} % provide euro and other symbols
\else % if luatex or xetex
  \usepackage{unicode-math} % this also loads fontspec
  \defaultfontfeatures{Scale=MatchLowercase}
  \defaultfontfeatures[\rmfamily]{Ligatures=TeX,Scale=1}
\fi
\usepackage{lmodern}
\ifPDFTeX\else
  % xetex/luatex font selection
\fi
% Use upquote if available, for straight quotes in verbatim environments
\IfFileExists{upquote.sty}{\usepackage{upquote}}{}
\IfFileExists{microtype.sty}{% use microtype if available
  \usepackage[]{microtype}
  \UseMicrotypeSet[protrusion]{basicmath} % disable protrusion for tt fonts
}{}
\makeatletter
\@ifundefined{KOMAClassName}{% if non-KOMA class
  \IfFileExists{parskip.sty}{%
    \usepackage{parskip}
  }{% else
    \setlength{\parindent}{0pt}
    \setlength{\parskip}{6pt plus 2pt minus 1pt}}
}{% if KOMA class
  \KOMAoptions{parskip=half}}
\makeatother
% Make \paragraph and \subparagraph free-standing
\makeatletter
\ifx\paragraph\undefined\else
  \let\oldparagraph\paragraph
  \renewcommand{\paragraph}{
    \@ifstar
      \xxxParagraphStar
      \xxxParagraphNoStar
  }
  \newcommand{\xxxParagraphStar}[1]{\oldparagraph*{#1}\mbox{}}
  \newcommand{\xxxParagraphNoStar}[1]{\oldparagraph{#1}\mbox{}}
\fi
\ifx\subparagraph\undefined\else
  \let\oldsubparagraph\subparagraph
  \renewcommand{\subparagraph}{
    \@ifstar
      \xxxSubParagraphStar
      \xxxSubParagraphNoStar
  }
  \newcommand{\xxxSubParagraphStar}[1]{\oldsubparagraph*{#1}\mbox{}}
  \newcommand{\xxxSubParagraphNoStar}[1]{\oldsubparagraph{#1}\mbox{}}
\fi
\makeatother

\usepackage{color}
\usepackage{fancyvrb}
\newcommand{\VerbBar}{|}
\newcommand{\VERB}{\Verb[commandchars=\\\{\}]}
\DefineVerbatimEnvironment{Highlighting}{Verbatim}{commandchars=\\\{\}}
% Add ',fontsize=\small' for more characters per line
\usepackage{framed}
\definecolor{shadecolor}{RGB}{241,243,245}
\newenvironment{Shaded}{\begin{snugshade}}{\end{snugshade}}
\newcommand{\AlertTok}[1]{\textcolor[rgb]{0.68,0.00,0.00}{#1}}
\newcommand{\AnnotationTok}[1]{\textcolor[rgb]{0.37,0.37,0.37}{#1}}
\newcommand{\AttributeTok}[1]{\textcolor[rgb]{0.40,0.45,0.13}{#1}}
\newcommand{\BaseNTok}[1]{\textcolor[rgb]{0.68,0.00,0.00}{#1}}
\newcommand{\BuiltInTok}[1]{\textcolor[rgb]{0.00,0.23,0.31}{#1}}
\newcommand{\CharTok}[1]{\textcolor[rgb]{0.13,0.47,0.30}{#1}}
\newcommand{\CommentTok}[1]{\textcolor[rgb]{0.37,0.37,0.37}{#1}}
\newcommand{\CommentVarTok}[1]{\textcolor[rgb]{0.37,0.37,0.37}{\textit{#1}}}
\newcommand{\ConstantTok}[1]{\textcolor[rgb]{0.56,0.35,0.01}{#1}}
\newcommand{\ControlFlowTok}[1]{\textcolor[rgb]{0.00,0.23,0.31}{\textbf{#1}}}
\newcommand{\DataTypeTok}[1]{\textcolor[rgb]{0.68,0.00,0.00}{#1}}
\newcommand{\DecValTok}[1]{\textcolor[rgb]{0.68,0.00,0.00}{#1}}
\newcommand{\DocumentationTok}[1]{\textcolor[rgb]{0.37,0.37,0.37}{\textit{#1}}}
\newcommand{\ErrorTok}[1]{\textcolor[rgb]{0.68,0.00,0.00}{#1}}
\newcommand{\ExtensionTok}[1]{\textcolor[rgb]{0.00,0.23,0.31}{#1}}
\newcommand{\FloatTok}[1]{\textcolor[rgb]{0.68,0.00,0.00}{#1}}
\newcommand{\FunctionTok}[1]{\textcolor[rgb]{0.28,0.35,0.67}{#1}}
\newcommand{\ImportTok}[1]{\textcolor[rgb]{0.00,0.46,0.62}{#1}}
\newcommand{\InformationTok}[1]{\textcolor[rgb]{0.37,0.37,0.37}{#1}}
\newcommand{\KeywordTok}[1]{\textcolor[rgb]{0.00,0.23,0.31}{\textbf{#1}}}
\newcommand{\NormalTok}[1]{\textcolor[rgb]{0.00,0.23,0.31}{#1}}
\newcommand{\OperatorTok}[1]{\textcolor[rgb]{0.37,0.37,0.37}{#1}}
\newcommand{\OtherTok}[1]{\textcolor[rgb]{0.00,0.23,0.31}{#1}}
\newcommand{\PreprocessorTok}[1]{\textcolor[rgb]{0.68,0.00,0.00}{#1}}
\newcommand{\RegionMarkerTok}[1]{\textcolor[rgb]{0.00,0.23,0.31}{#1}}
\newcommand{\SpecialCharTok}[1]{\textcolor[rgb]{0.37,0.37,0.37}{#1}}
\newcommand{\SpecialStringTok}[1]{\textcolor[rgb]{0.13,0.47,0.30}{#1}}
\newcommand{\StringTok}[1]{\textcolor[rgb]{0.13,0.47,0.30}{#1}}
\newcommand{\VariableTok}[1]{\textcolor[rgb]{0.07,0.07,0.07}{#1}}
\newcommand{\VerbatimStringTok}[1]{\textcolor[rgb]{0.13,0.47,0.30}{#1}}
\newcommand{\WarningTok}[1]{\textcolor[rgb]{0.37,0.37,0.37}{\textit{#1}}}

\usepackage{longtable,booktabs,array}
\usepackage{calc} % for calculating minipage widths
% Correct order of tables after \paragraph or \subparagraph
\usepackage{etoolbox}
\makeatletter
\patchcmd\longtable{\par}{\if@noskipsec\mbox{}\fi\par}{}{}
\makeatother
% Allow footnotes in longtable head/foot
\IfFileExists{footnotehyper.sty}{\usepackage{footnotehyper}}{\usepackage{footnote}}
\makesavenoteenv{longtable}
\usepackage{graphicx}
\makeatletter
\newsavebox\pandoc@box
\newcommand*\pandocbounded[1]{% scales image to fit in text height/width
  \sbox\pandoc@box{#1}%
  \Gscale@div\@tempa{\textheight}{\dimexpr\ht\pandoc@box+\dp\pandoc@box\relax}%
  \Gscale@div\@tempb{\linewidth}{\wd\pandoc@box}%
  \ifdim\@tempb\p@<\@tempa\p@\let\@tempa\@tempb\fi% select the smaller of both
  \ifdim\@tempa\p@<\p@\scalebox{\@tempa}{\usebox\pandoc@box}%
  \else\usebox{\pandoc@box}%
  \fi%
}
% Set default figure placement to htbp
\def\fps@figure{htbp}
\makeatother

\ifLuaTeX
  \usepackage{luacolor}
  \usepackage[soul]{lua-ul}
\else
  \usepackage{soul}
\fi




\setlength{\emergencystretch}{3em} % prevent overfull lines

\providecommand{\tightlist}{%
  \setlength{\itemsep}{0pt}\setlength{\parskip}{0pt}}



 


\KOMAoption{captions}{tableheading}
\makeatletter
\@ifpackageloaded{tcolorbox}{}{\usepackage[skins,breakable]{tcolorbox}}
\@ifpackageloaded{fontawesome5}{}{\usepackage{fontawesome5}}
\definecolor{quarto-callout-color}{HTML}{909090}
\definecolor{quarto-callout-note-color}{HTML}{0758E5}
\definecolor{quarto-callout-important-color}{HTML}{CC1914}
\definecolor{quarto-callout-warning-color}{HTML}{EB9113}
\definecolor{quarto-callout-tip-color}{HTML}{00A047}
\definecolor{quarto-callout-caution-color}{HTML}{FC5300}
\definecolor{quarto-callout-color-frame}{HTML}{acacac}
\definecolor{quarto-callout-note-color-frame}{HTML}{4582ec}
\definecolor{quarto-callout-important-color-frame}{HTML}{d9534f}
\definecolor{quarto-callout-warning-color-frame}{HTML}{f0ad4e}
\definecolor{quarto-callout-tip-color-frame}{HTML}{02b875}
\definecolor{quarto-callout-caution-color-frame}{HTML}{fd7e14}
\makeatother
\makeatletter
\@ifpackageloaded{caption}{}{\usepackage{caption}}
\AtBeginDocument{%
\ifdefined\contentsname
  \renewcommand*\contentsname{Table of contents}
\else
  \newcommand\contentsname{Table of contents}
\fi
\ifdefined\listfigurename
  \renewcommand*\listfigurename{List of Figures}
\else
  \newcommand\listfigurename{List of Figures}
\fi
\ifdefined\listtablename
  \renewcommand*\listtablename{List of Tables}
\else
  \newcommand\listtablename{List of Tables}
\fi
\ifdefined\figurename
  \renewcommand*\figurename{Figure}
\else
  \newcommand\figurename{Figure}
\fi
\ifdefined\tablename
  \renewcommand*\tablename{Table}
\else
  \newcommand\tablename{Table}
\fi
}
\@ifpackageloaded{float}{}{\usepackage{float}}
\floatstyle{ruled}
\@ifundefined{c@chapter}{\newfloat{codelisting}{h}{lop}}{\newfloat{codelisting}{h}{lop}[chapter]}
\floatname{codelisting}{Listing}
\newcommand*\listoflistings{\listof{codelisting}{List of Listings}}
\makeatother
\makeatletter
\makeatother
\makeatletter
\@ifpackageloaded{caption}{}{\usepackage{caption}}
\@ifpackageloaded{subcaption}{}{\usepackage{subcaption}}
\makeatother
\usepackage{bookmark}
\IfFileExists{xurl.sty}{\usepackage{xurl}}{} % add URL line breaks if available
\urlstyle{same}
\hypersetup{
  pdftitle={Exam 2 review},
  colorlinks=true,
  linkcolor={blue},
  filecolor={Maroon},
  citecolor={Blue},
  urlcolor={Blue},
  pdfcreator={LaTeX via pandoc}}


\title{Exam 2 review}
\usepackage{etoolbox}
\makeatletter
\providecommand{\subtitle}[1]{% add subtitle to \maketitle
  \apptocmd{\@title}{\par {\large #1 \par}}{}{}
}
\makeatother
\subtitle{Questions}
\author{}
\date{}
\begin{document}
\maketitle

\renewcommand*\contentsname{Table of contents}
{
\hypersetup{linkcolor=}
\setcounter{tocdepth}{3}
\tableofcontents
}

\begin{longtable}[]{@{}
  >{\raggedright\arraybackslash}p{(\linewidth - 4\tabcolsep) * \real{0.1737}}
  >{\raggedright\arraybackslash}p{(\linewidth - 4\tabcolsep) * \real{0.7545}}
  >{\raggedright\arraybackslash}p{(\linewidth - 4\tabcolsep) * \real{0.0659}}@{}}
\toprule\noalign{}
\endhead
\bottomrule\noalign{}
\endlastfoot
Course GitHub organization & 🔗 on
\href{https://github.com/sta199-f25}{GitHub} & \\
RStudio containers & 🔗 on
\href{https://cmgr.oit.duke.edu/containers}{Duke Container Manager} & \\
Office hours & 🔗 on
\href{https://docs.google.com/spreadsheets/d/1rVx9uZVNcrG_nUSDo7iol5VEeGcHURASVtBpwZjz5bc/edit?usp=sharing}{Google
Sheets} & \\
Ed Discussion & 🔗 on
\href{https://edstem.org/us/courses/80616/discussion/}{Canvas} & \\
Gradescope & 🔗 on
\href{https://www.gradescope.com/courses/1067800}{Canvas} & \\
Gradebook & 🔗 on
\href{https://canvas.duke.edu/courses/61539/gradebook}{Canvas} & \\
Texbooks & 🔗 \href{https://r4ds.hadley.nz/}{R for Data Science}

🔗 \href{https://openintro-ims2.netlify.app/}{Introduction to Modern
Statistics} & \\
Package documentation & 🔗 ggplot2:
\href{https://ggplot2.tidyverse.org/}{ggplot2.tidyverse.org}

🔗 dplyr: \href{https://dplyr.tidyverse.org/}{dplyr.tidyverse.org}

🔗 tidyr: \href{https://tidyr.tidyverse.org/}{tidyr.tidyverse.org}

🔗 forcats: \href{https://forcats.tidyverse.org/}{forcats.tidyverse.org}

🔗 stringr: \href{https://stringr.tidyverse.org/}{stringr.tidyverse.org}

🔗 lubridate:
\href{https://lubridate.tidyverse.org/}{lubridate.tidyverse.org}

🔗 readr: \href{https://readr.tidyverse.org/}{readr.tidyverse.org}

🔗 readxl: \href{https://readxl.tidyverse.org/}{readxl.tidyverse.org}
& \\
\end{longtable}

\begin{Shaded}
\begin{Highlighting}[]
\NormalTok{\#| label: load{-}packages}
\NormalTok{\#| message: false}
\NormalTok{library(tidyverse)}
\NormalTok{library(tidymodels)}
\NormalTok{library(openintro)}
\NormalTok{library(scales)}
\NormalTok{theme\_set(theme\_minimal(base\_size = 12))}
\end{Highlighting}
\end{Shaded}

\begin{Shaded}
\begin{Highlighting}[]
\NormalTok{\#| label: blizzard{-}data{-}prep}
\NormalTok{blizzard\_salary \textless{}{-} blizzard\_salary |\textgreater{}}
\NormalTok{  mutate(}
\NormalTok{    annual\_salary = case\_when(}
\NormalTok{      salary\_type == "week" \textasciitilde{} current\_salary * 52,}
\NormalTok{      salary\_type == "hour" \textasciitilde{} current\_salary * 40 * 52,}
\NormalTok{      TRUE \textasciitilde{} current\_salary}
\NormalTok{    ),}
\NormalTok{    performance\_rating = if\_else(}
\NormalTok{      performance\_rating == "Developing",}
\NormalTok{      "Poor",}
\NormalTok{      performance\_rating}
\NormalTok{    )}
\NormalTok{  ) |\textgreater{}}
\NormalTok{  filter(salary\_type != "week") |\textgreater{}}
\NormalTok{  mutate(}
\NormalTok{    salary\_type = if\_else(salary\_type == "hour", "Hourly", "Salaried")}
\NormalTok{  ) |\textgreater{}}
\NormalTok{  filter(!is.na(annual\_salary)) |\textgreater{}}
\NormalTok{  select(percent\_incr, salary\_type, annual\_salary, performance\_rating)}
\end{Highlighting}
\end{Shaded}

\begin{tcolorbox}[enhanced jigsaw, breakable, coltitle=black, opacityback=0, colbacktitle=quarto-callout-note-color!10!white, colback=white, titlerule=0mm, leftrule=.75mm, left=2mm, opacitybacktitle=0.6, toprule=.15mm, toptitle=1mm, colframe=quarto-callout-note-color-frame, bottomtitle=1mm, bottomrule=.15mm, title=\textcolor{quarto-callout-note-color}{\faInfo}\hspace{0.5em}{Note}, arc=.35mm, rightrule=.15mm]

Suggested answers can be found
\href{./exam-review/exam-2-review-A.qmd}{here}, but resist the urge to
peek before you go through it yourself.

\end{tcolorbox}

\subsection{Blizzard salaries}\label{blizzard-salaries}

In 2020, employees of Blizzard Entertainment circulated a spreadsheet to
anonymously share salaries and recent pay increases amidst rising
tension in the video game industry over wage disparities and executive
compensation. (Source:
\href{https://www.bloomberg.com/news/articles/2020-08-03/blizzard-workers-share-salaries-in-revolt-over-wage-disparities}{Blizzard
Workers Share Salaries in Revolt Over Pay})

The name of the data frame used for this analysis is
\texttt{blizzard\_salary} and the variables are:

\begin{itemize}
\item
  \texttt{percent\_incr}: Raise given in July 2020, as percent increase
  with values ranging from 1 (1\% increase) to 21.5 (21.5\% increase)
\item
  \texttt{salary\_type}: Type of salary, with levels \texttt{Hourly} and
  \texttt{Salaried}
\item
  \texttt{annual\_salary}: Annual salary, in USD, with values ranging
  from \$50,939 to \$216,856.
\item
  \texttt{performance\_rating}: Most recent review performance rating,
  with levels \texttt{Poor}, \texttt{Successful}, \texttt{High}, and
  \texttt{Top}. The \texttt{Poor} level is the lowest rating and the
  \texttt{Top} level is the highest rating.
\end{itemize}

The first ten rows of \texttt{blizzard\_salary} are shown below:

\begin{Shaded}
\begin{Highlighting}[]
\NormalTok{blizzard\_salary |\textgreater{}}
\NormalTok{  select(percent\_incr, salary\_type, annual\_salary, performance\_rating)}
\end{Highlighting}
\end{Shaded}

\subsection{Question 1}\label{question-1}

You fit a model for predicting raises (\texttt{percent\_incr}) from
salaries (\texttt{annual\_salary}). We'll call this model
\texttt{raise\_1\_fit}. A tidy output of the model is shown below.

\begin{Shaded}
\begin{Highlighting}[]
\NormalTok{\#| label: raise{-}salary{-}fit}
\NormalTok{raise\_1\_fit \textless{}{-} linear\_reg() |\textgreater{}}
\NormalTok{  fit(percent\_incr \textasciitilde{} annual\_salary, data = blizzard\_salary)}

\NormalTok{tidy(raise\_1\_fit)}
\end{Highlighting}
\end{Shaded}

Which of the following is the best interpretation of the slope
coefficient?

\begin{enumerate}
\def\labelenumi{\alph{enumi}.}
\tightlist
\item
  For every additional \$1,000 of annual salary, the model predicts the
  raise to be higher, on average, by 1.55\%.
\item
  For every additional \$1,000 of annual salary, the raise goes up by
  0.0155\%.
\item
  For every additional \$1,000 of annual salary, the model predicts the
  raise to be higher, on average, by 0.0155\%.
\item
  For every additional \$1,000 of annual salary, the model predicts the
  raise to be higher, on average, by 1.87\%.
\end{enumerate}

\subsection{Question 2}\label{question-2}

You then fit a model for predicting raises (\texttt{percent\_incr}) from
salaries (\texttt{annual\_salary}) and performance ratings
(\texttt{performance\_rating}). We'll call this model
\texttt{raise\_2\_fit}. Which of the following is definitely true based
on the information you have so far?

\begin{enumerate}
\def\labelenumi{\alph{enumi}.}
\tightlist
\item
  Intercept of \texttt{raise\_2\_fit} is higher than intercept of
  \texttt{raise\_1\_fit}.
\item
  Slope of \texttt{raise\_2\_fit} is higher than RMSE of
  \texttt{raise\_1\_fit}.
\item
  Adjusted \(R^2\) of \texttt{raise\_2\_fit} is higher than adjusted
  \(R^2\) of \texttt{raise\_1\_fit}.
\item
  \(R^2\) of \texttt{raise\_2\_fit} is higher \(R^2\) of
  \texttt{raise\_1\_fit}.
\end{enumerate}

\newpage{}

\subsection{Question 3}\label{question-3}

The tidy model output for the \texttt{raise\_2\_fit} model you fit is
shown below.

\begin{Shaded}
\begin{Highlighting}[]
\NormalTok{\#| label: raise{-}salary{-}rating{-}fit}
\NormalTok{raise\_2\_fit \textless{}{-} linear\_reg() |\textgreater{}}
\NormalTok{  fit(percent\_incr \textasciitilde{} annual\_salary + performance\_rating, data = blizzard\_salary)}

\NormalTok{tidy(raise\_2\_fit)}
\end{Highlighting}
\end{Shaded}

When your teammate sees this model output, they remark ``The coefficient
for \texttt{performance\_ratingSuccessful} is negative. That's weird. I
guess it means that people who get successful performance ratings get
lower raises.'' How would you respond to your teammate?

\(\vspace{1.5cm}\)

\subsection{Question 4}\label{question-4}

Ultimately, your teammate decides they don't like the negative slope
coefficients in the model output you created (not that there's anything
wrong with negative slope coefficients!), does something else, and comes
up with the following model output.

\begin{Shaded}
\begin{Highlighting}[]
\NormalTok{blizzard\_salary \textless{}{-} blizzard\_salary |\textgreater{}}
\NormalTok{  mutate(}
\NormalTok{    performance\_rating = fct\_relevel(}
\NormalTok{      performance\_rating,}
\NormalTok{      "Poor",}
\NormalTok{      "Successful",}
\NormalTok{      "High",}
\NormalTok{      "Top"}
\NormalTok{    )}
\NormalTok{  )}

\NormalTok{raise\_2\_fit \textless{}{-} linear\_reg() |\textgreater{}}
\NormalTok{  fit(percent\_incr \textasciitilde{} annual\_salary + performance\_rating, data = blizzard\_salary)}

\NormalTok{tidy(raise\_2\_fit)}
\end{Highlighting}
\end{Shaded}

Unfortunately they didn't write their code in a Quarto document, instead
just wrote some code in the Console and then lost track of their work.
They remember using the \texttt{fct\_relevel()} function and doing
something like the following:

\begin{Shaded}
\begin{Highlighting}[]
\NormalTok{\#| eval: false}
\NormalTok{\#| echo: true}
\NormalTok{blizzard\_salary \textless{}{-} blizzard\_salary |\textgreater{}}
\NormalTok{  mutate(performance\_rating = fct\_relevel(performance\_rating, \_\_\_))}
\end{Highlighting}
\end{Shaded}

What should they put in the blanks to get the same model output as
above?

\begin{enumerate}
\def\labelenumi{\alph{enumi}.}
\tightlist
\item
  ``Poor'', ``Successful'', ``High'', ``Top''
\item
  ``Successful'', ``High'', ``Top''
\item
  ``Top'', ``High'', ``Successful'', ``Poor''
\item
  Poor, Successful, High, Top
\end{enumerate}

\subsection{Question 5}\label{question-5}

Suppose we fit a model to predict \texttt{percent\_incr} from
\texttt{annual\_salary} and \texttt{salary\_type}. A tidy output of the
model is shown below.

\begin{Shaded}
\begin{Highlighting}[]
\NormalTok{\#| label: raise{-}salary{-}type{-}fit}
\NormalTok{raise\_3\_fit \textless{}{-} linear\_reg() |\textgreater{}}
\NormalTok{  fit(percent\_incr \textasciitilde{} annual\_salary + salary\_type, data = blizzard\_salary)}

\NormalTok{tidy(raise\_3\_fit)}
\end{Highlighting}
\end{Shaded}

Which of the following visualizations represent this model? Explain your
reasoning.

\begin{Shaded}
\begin{Highlighting}[]
\NormalTok{\#| label: fig{-}raise{-}salary{-}type}
\NormalTok{\#| warning: false}
\NormalTok{\#| layout{-}ncol: 2}
\NormalTok{\#| fig{-}cap: |}
\NormalTok{\#|   Visualizations of the relationship between percent increase, annual}
\NormalTok{\#|   salary, and salary type}
\NormalTok{\#| fig{-}subcap:}
\NormalTok{\#|   {-} Option 1}
\NormalTok{\#|   {-} Option 2}
\NormalTok{\#|   {-} Option 3}
\NormalTok{\#|   {-} Option 4}
\NormalTok{ggplot(}
\NormalTok{  blizzard\_salary,}
\NormalTok{  aes(x = annual\_salary, y = percent\_incr, color = salary\_type)}
\NormalTok{) +}
\NormalTok{  geom\_point(aes(shape = salary\_type), alpha = 0.5, size = 2) +}
\NormalTok{  geom\_smooth(}
\NormalTok{    aes(linetype = salary\_type),}
\NormalTok{    method = "lm",}
\NormalTok{    se = FALSE,}
\NormalTok{    fullrange = TRUE,}
\NormalTok{    linewidth = 1.5}
\NormalTok{  ) +}
\NormalTok{  labs(}
\NormalTok{    x = "Annual salary",}
\NormalTok{    y = "Percent increase",}
\NormalTok{    color = "Salary type",}
\NormalTok{    linetype = "Salary type",}
\NormalTok{    shape = "Salary type"}
\NormalTok{  ) +}
\NormalTok{  scale\_x\_continuous(labels = label\_dollar()) +}
\NormalTok{  scale\_y\_continuous(labels = label\_percent(scale = 1)) +}
\NormalTok{  theme(legend.position = "top")}

\NormalTok{ggplot(}
\NormalTok{  blizzard\_salary,}
\NormalTok{  aes(x = annual\_salary, y = percent\_incr, color = salary\_type)}
\NormalTok{) +}
\NormalTok{  geom\_point(aes(shape = salary\_type), alpha = 0.5, size = 2) +}
\NormalTok{  geom\_smooth(}
\NormalTok{    aes(linetype = salary\_type),}
\NormalTok{    se = FALSE,}
\NormalTok{    fullrange = TRUE,}
\NormalTok{    linewidth = 1.5}
\NormalTok{  ) +}
\NormalTok{  labs(}
\NormalTok{    x = "Annual salary",}
\NormalTok{    y = "Percent increase"}
\NormalTok{  ) +}
\NormalTok{  scale\_x\_continuous(labels = label\_dollar()) +}
\NormalTok{  scale\_y\_continuous(labels = label\_percent(scale = 1)) +}
\NormalTok{  theme(legend.position = "top")}

\NormalTok{ggplot(}
\NormalTok{  blizzard\_salary,}
\NormalTok{  aes(x = annual\_salary, y = percent\_incr, color = salary\_type)}
\NormalTok{) +}
\NormalTok{  geom\_point(}
\NormalTok{    aes(shape = salary\_type),}
\NormalTok{    alpha = 0.5,}
\NormalTok{    size = 2,}
\NormalTok{    show.legend = FALSE}
\NormalTok{  ) +}
\NormalTok{  geom\_abline(}
\NormalTok{    intercept = 1.24,}
\NormalTok{    slope = 0.0000137,}
\NormalTok{    color = "\#E87d72",}
\NormalTok{    linewidth = 1.5,}
\NormalTok{    linetype = "solid"}
\NormalTok{  ) +}
\NormalTok{  geom\_abline(}
\NormalTok{    intercept = 1.24 + 0.913,}
\NormalTok{    slope = 0.0000137,}
\NormalTok{    color = "\#56bcc2",}
\NormalTok{    linewidth = 1.5,}
\NormalTok{    linetype = "dashed"}
\NormalTok{  ) +}
\NormalTok{  labs(}
\NormalTok{    x = "Annual salary",}
\NormalTok{    y = "Percent increase",}
\NormalTok{  ) +}
\NormalTok{  scale\_x\_continuous(labels = label\_dollar()) +}
\NormalTok{  scale\_y\_continuous(labels = label\_percent(scale = 1))}

\NormalTok{ggplot(}
\NormalTok{  blizzard\_salary,}
\NormalTok{  aes(x = annual\_salary, y = percent\_incr, color = salary\_type)}
\NormalTok{) +}
\NormalTok{  geom\_point(alpha = 0.5, size = 2, show.legend = FALSE) +}
\NormalTok{  geom\_abline(}
\NormalTok{    intercept = 1.24,}
\NormalTok{    slope = 0.0000137,}
\NormalTok{    color = "\#56bcc2",}
\NormalTok{    linewidth = 1.5,}
\NormalTok{    linetype = "dashed"}
\NormalTok{  ) +}
\NormalTok{  geom\_abline(}
\NormalTok{    intercept = 1.24 + 0.913,}
\NormalTok{    slope = 0.0000137,}
\NormalTok{    color = "\#E87d72",}
\NormalTok{    linewidth = 1.5}
\NormalTok{  ) +}
\NormalTok{  labs(}
\NormalTok{    x = "Annual salary",}
\NormalTok{    y = "Percent increase",}
\NormalTok{  ) +}
\NormalTok{  scale\_x\_continuous(labels = label\_dollar()) +}
\NormalTok{  scale\_y\_continuous(labels = label\_percent(scale = 1))}
\end{Highlighting}
\end{Shaded}

\newpage{}

\newpage{}

\subsection{Question 6}\label{question-6}

Suppose you now fit a model to predict the natural log of percent
increase, \texttt{log(percent\_incr)}, from performance rating. The
model is called \texttt{raise\_4\_fit}.

\begin{Shaded}
\begin{Highlighting}[]
\NormalTok{raise\_4\_fit \textless{}{-} linear\_reg() |\textgreater{}}
\NormalTok{  fit(log(percent\_incr + 0.0001) \textasciitilde{} performance\_rating, data = blizzard\_salary)}
\end{Highlighting}
\end{Shaded}

You're provided the following:

\begin{Shaded}
\begin{Highlighting}[]
\NormalTok{\#| echo: true}
\NormalTok{tidy(raise\_4\_fit) |\textgreater{}}
\NormalTok{  select(term, estimate) |\textgreater{}}
\NormalTok{  mutate(exp\_estimate = exp(estimate))}
\end{Highlighting}
\end{Shaded}

Based on this, which of the following is true?

a. The model predicts that the percentage increase employees with
Successful performance get, on average, is higher by 10.25\% compared to
the employees with Poor performance rating.

b. The model predicts that the percentage increase employees with
Successful performance get, on average, is higher by 6.93\% compared to
the employees with Poor performance rating.

c. The model predicts that the percentage increase employees with
Successful performance get, on average, is higher by a factor of 1025
compared to the employees with Poor performance rating.

d. The model predicts that the percentage increase employees with
Successful performance get, on average, is higher by a factor of 6.93
compared to the employees with Poor performance rating.

\newpage{}

\section{Movies}\label{movies}

The data for this part comes from the Internet Movie Database (IMDB).
Specifically, the data are a random sample of movies released between
1980 and 2020.

\begin{Shaded}
\begin{Highlighting}[]
\NormalTok{\#| label: load{-}data}
\NormalTok{\#| message: false}
\NormalTok{movies \textless{}{-} read\_csv("data/movies.csv")}
\end{Highlighting}
\end{Shaded}

The name of the data frame used for this analysis is \texttt{movies},
and it contains the variables shown in Table~\ref{tbl-data-dictionary}.

\setcounter{table}{0}

\begin{longtable}[]{@{}
  >{\raggedright\arraybackslash}p{(\linewidth - 2\tabcolsep) * \real{0.2000}}
  >{\raggedright\arraybackslash}p{(\linewidth - 2\tabcolsep) * \real{0.8000}}@{}}
\caption{Data dictionary for
\texttt{movies}}\label{tbl-data-dictionary}\tabularnewline
\toprule\noalign{}
\begin{minipage}[b]{\linewidth}\raggedright
Variable
\end{minipage} & \begin{minipage}[b]{\linewidth}\raggedright
Description
\end{minipage} \\
\midrule\noalign{}
\endfirsthead
\toprule\noalign{}
\begin{minipage}[b]{\linewidth}\raggedright
Variable
\end{minipage} & \begin{minipage}[b]{\linewidth}\raggedright
Description
\end{minipage} \\
\midrule\noalign{}
\endhead
\bottomrule\noalign{}
\endlastfoot
\texttt{name} & name of the movie \\
\texttt{rating} & rating of the movie (R, PG, etc.) \\
\texttt{genre} & main genre of the movie. \\
\texttt{runtime} & duration of the movie \\
\texttt{year} & year of release \\
\texttt{release\_date} & release date (YYYY-MM-DD) \\
\texttt{release\_country} & release country \\
\texttt{score} & IMDB user rating \\
\texttt{votes} & number of user votes \\
\texttt{director} & the director \\
\texttt{writer} & writer of the movie \\
\texttt{star} & main actor/actress \\
\texttt{country} & country of origin \\
\texttt{budget} & the budget of a movie (some movies don't have this, so
it appears as 0) \\
\texttt{gross} & revenue of the movie \\
\texttt{company} & the production company \\
\end{longtable}

The first thirty rows of the \texttt{movies} data frame are shown in
Table~\ref{tbl-data}, with variable types suppressed (since we'll ask
about them later).

\newpage{}

\begin{Shaded}
\begin{Highlighting}[]
\NormalTok{\#| echo: false}
\NormalTok{movies\_to\_mark \textless{}{-} c("Blue City", "Rang De Basanti", "Winter Sleep")}

\NormalTok{movies \textless{}{-} movies |\textgreater{}}
\NormalTok{  mutate(}
\NormalTok{    mark = if\_else(name \%in\% movies\_to\_mark, TRUE, FALSE),}
\NormalTok{    rating = case\_when(}
\NormalTok{      rating == "TV{-}PG" \textasciitilde{} "PG",}
\NormalTok{      rating == "Unrated" \textasciitilde{} "Not Rated",}
\NormalTok{      is.na(rating) \textasciitilde{} "Not Rated",}
\NormalTok{      .default = rating}
\NormalTok{    ),}
\NormalTok{    rating = fct\_relevel(rating, "G", "PG", "PG{-}13", "R", "NC{-}17", "Not Rated")}
\NormalTok{  ) |\textgreater{}}
\NormalTok{  arrange(desc(mark)) |\textgreater{}}
\NormalTok{  relocate(}
\NormalTok{    name,}
\NormalTok{    score,}
\NormalTok{    runtime,}
\NormalTok{    genre,}
\NormalTok{    rating,}
\NormalTok{    release\_country,}
\NormalTok{    release\_date,}
\NormalTok{    budget,}
\NormalTok{    gross,}
\NormalTok{    votes,}
\NormalTok{    year,}
\NormalTok{    director,}
\NormalTok{    writer,}
\NormalTok{    star,}
\NormalTok{    company,}
\NormalTok{    country}
\NormalTok{  )}
\end{Highlighting}
\end{Shaded}

\begin{landscape}

\begin{table}

\caption{\label{tbl-data}}

\centering{

First 30 rows of \texttt{movies}, with variable types suppressed.

\begin{Shaded}
\begin{Highlighting}[]
\NormalTok{\#| echo: false}
\NormalTok{options(}
\NormalTok{  dplyr.print\_min = 30,}
\NormalTok{  pillar.min\_chars = 13,}
\NormalTok{  pillar.width = 110,}
\NormalTok{  pillar.sigfig = 6}
\NormalTok{)}

\NormalTok{format(movies |\textgreater{} select(!c(mark)))[{-}3L] |\textgreater{}}
\NormalTok{  str\_remove\_all(" \textless{}.*?\textgreater{}") |\textgreater{}}
\NormalTok{  cat(sep = "\textbackslash{}n")}

\NormalTok{options(}
\NormalTok{  dplyr.print\_min = 10,}
\NormalTok{  pillar.min\_chars = 8,}
\NormalTok{  pillar.width = 80,}
\NormalTok{  pillar.sigfig = 3}
\NormalTok{)}
\end{Highlighting}
\end{Shaded}

}

\end{table}%

\end{landscape}
\clearpage

\newpage{}

\newpage{}

\subsection{Score vs.~runtime}\label{score-vs.-runtime}

In this part, we fit a model predicting \texttt{score} from
\texttt{runtime} and name it \texttt{score\_runtime\_fit}.

\begin{Shaded}
\begin{Highlighting}[]
\NormalTok{\#| echo: true}
\NormalTok{score\_runtime\_fit \textless{}{-} linear\_reg() |\textgreater{}}
\NormalTok{  fit(score \textasciitilde{} runtime, data = movies)}
\end{Highlighting}
\end{Shaded}

\textbf{?@fig-score-runtime} visualizes the relationship between
\texttt{score} and \texttt{runtime} as well as the linear model for
predicting \texttt{score} from \texttt{runtime}. The first three movies
in Table~\ref{tbl-data} are labeled in the visualization as well. Answer
all questions in this part based on \textbf{?@fig-score-runtime}.

\begin{Shaded}
\begin{Highlighting}[]
\NormalTok{\#| label: fig{-}score{-}runtime{-}prep}
\NormalTok{\#| echo: false}
\NormalTok{movies \textless{}{-} movies |\textgreater{}}
\NormalTok{  mutate(}
\NormalTok{    runtime = parse\_number(runtime),}
\NormalTok{    shape1 = case\_when(}
\NormalTok{      name == movies\_to\_mark[1] \textasciitilde{} "circle",}
\NormalTok{      name == movies\_to\_mark[2] \textasciitilde{} "square",}
\NormalTok{      name == movies\_to\_mark[3] \textasciitilde{} "triangle",}
\NormalTok{      .default = "circle"}
\NormalTok{    ),}
\NormalTok{    shape2 = case\_when(}
\NormalTok{      name == movies\_to\_mark[1] \textasciitilde{} "circle open",}
\NormalTok{      name == movies\_to\_mark[2] \textasciitilde{} "square open",}
\NormalTok{      name == movies\_to\_mark[3] \textasciitilde{} "triangle open",}
\NormalTok{      .default = NA}
\NormalTok{    ),}
\NormalTok{    color = if\_else(mark, "black", "gray50"),}
\NormalTok{    alpha = if\_else(mark, 1, 0.5)}
\NormalTok{  )}
\end{Highlighting}
\end{Shaded}

\begin{Shaded}
\begin{Highlighting}[]
\NormalTok{\#| label: fig{-}score{-}runtime}
\NormalTok{\#| fig{-}cap: Scatterplot of \textasciigrave{}score\textasciigrave{} vs. \textasciigrave{}runtime\textasciigrave{} for \textasciigrave{}movies\textasciigrave{}.}
\NormalTok{\#| echo: false}
\NormalTok{\#| message: false}
\NormalTok{\#| fig{-}asp: 0.5}
\NormalTok{\#| fig{-}width: 6.5}
\NormalTok{ggplot(movies, aes(x = runtime, y = score)) +}
\NormalTok{  geom\_point(aes(shape = shape1, color = color, alpha = alpha)) +}
\NormalTok{  geom\_smooth(method = "lm", se = FALSE, color = "black") +}
\NormalTok{  scale\_shape\_identity() +}
\NormalTok{  scale\_alpha\_identity() +}
\NormalTok{  scale\_color\_identity() +}
\NormalTok{  geom\_point(}
\NormalTok{    data = movies |\textgreater{} filter(mark),}
\NormalTok{    aes(shape = shape2),}
\NormalTok{    size = 3}
\NormalTok{  ) +}
\NormalTok{  geom\_label(}
\NormalTok{    data = movies |\textgreater{} filter(mark),}
\NormalTok{    aes(label = name),}
\NormalTok{    size = 3,}
\NormalTok{    nudge\_x = 1,}
\NormalTok{    nudge\_y = c({-}0.4, {-}0.4, 0.4)}
\NormalTok{  ) +}
\NormalTok{  scale\_x\_continuous(breaks = seq(80, 240, 20))}
\end{Highlighting}
\end{Shaded}

\newpage{}

\subsubsection{Question 7}\label{question-7}

Partial code for producing \textbf{?@fig-score-runtime} is given below.
Which of the following goes in the blank on Line 2? \textbf{Select all
that apply.}

\begin{Shaded}
\begin{Highlighting}[]
\NormalTok{\#| eval: false}
\NormalTok{\#| echo: true}
\NormalTok{\#| code{-}line{-}numbers: true}
\NormalTok{movies |\textgreater{}}
\NormalTok{  mutate(runtime = \_\_\_) |\textgreater{}}
\NormalTok{  ggplot(aes(x = runtime, y = score)) +}
\NormalTok{  geom\_point(alpha = 0.5) +}
\NormalTok{  geom\_smooth(method = "lm", se = FALSE)}
\NormalTok{\# additional code for annotating Blue City on the plot}
\end{Highlighting}
\end{Shaded}

\begin{enumerate}
\def\labelenumi{\alph{enumi}.}
\item
  \texttt{grepl("\ mins",\ runtime)}
\item
  \texttt{grep("\ mins",\ runtime)}
\item
  \texttt{str\_remove(runtime,\ "\ mins")}
\item
  \texttt{as.numeric(str\_remove(runtime,\ "\ mins"))}
\item
  \texttt{na.rm(runtime)}
\end{enumerate}

\subsubsection{Question 8}\label{question-8}

Based on this model, order the three labeled movies in
\textbf{?@fig-score-runtime} in decreasing order of the magnitude
(absolute value) of their residuals.

\begin{enumerate}
\def\labelenumi{\alph{enumi}.}
\item
  Winter Sleep \textgreater{} Rang De Basanti \textgreater{} Blue City
\item
  Winter Sleep \textgreater{} Blue City \textgreater{} Rang De Basanti
\item
  Rang De Basanti \textgreater{} Winter Sleep \textgreater{} Blue City
\item
  Blue City \textgreater{} Winter Sleep \textgreater{} Rang De Basanti
\item
  Blue City \textgreater{} Rang De Basanti \textgreater{} Winter Sleep
\end{enumerate}

\newpage{}

\subsubsection{Question 9}\label{question-9}

\begin{Shaded}
\begin{Highlighting}[]
\NormalTok{\#| include: false}
\NormalTok{score\_runtime\_rsq \textless{}{-} round(glance(score\_runtime\_fit)$r.squared * 100, 0)}
\end{Highlighting}
\end{Shaded}

The R-squared for the model visualized in \textbf{?@fig-score-runtime}
is \texttt{\{r\}\ score\_runtime\_rsq}\%. Which of the following is the
\ul{\textbf{best}} interpretation of this value?

\begin{enumerate}
\def\labelenumi{\alph{enumi}.}
\item
  \texttt{\{r\}\ score\_runtime\_rsq}\% of the variability in movie
  runtimes is explained by their scores.
\item
  \texttt{\{r\}\ score\_runtime\_rsq}\% of the variability in movie
  scores is explained by their runtime.
\item
  The model accurately predicts scores of
  \texttt{\{r\}\ score\_runtime\_rsq}\% of the movies in this sample.
\item
  The model accurately predicts scores of
  \texttt{\{r\}\ score\_runtime\_rsq}\% of all movies.
\item
  The correlation between scores and runtimes of movies is
  \texttt{\{r\}\ score\_runtime\_rsq\ /\ 100}.
\end{enumerate}

\subsection{Score vs.~runtime or year}\label{score-vs.-runtime-or-year}

The visualizations below show the relationship between \texttt{score}
and \texttt{runtime} as well as \texttt{score} and \texttt{year},
respectively. Additionally, the lines of best fit are overlaid on the
visualizations.

\begin{Shaded}
\begin{Highlighting}[]
\NormalTok{\#| layout{-}ncol: 2}
\NormalTok{\#| echo: false}
\NormalTok{\#| message: false}
\NormalTok{ggplot(movies, aes(x = runtime, y = score)) +}
\NormalTok{  geom\_point(alpha = 0.5, color = "gray40") +}
\NormalTok{  geom\_smooth(method = "lm", se = FALSE, color = "black", linewidth = 1.2)}

\NormalTok{ggplot(movies, aes(x = year, y = score)) +}
\NormalTok{  geom\_point(alpha = 0.5, color = "gray40") +}
\NormalTok{  geom\_smooth(method = "lm", se = FALSE, color = "black", linewidth = 1.2)}
\end{Highlighting}
\end{Shaded}

The correlation coefficients of these relationships are calculated
below, though some of the code and the output are missing. Answer all
questions in this part based on the code and output shown below.

\begin{Shaded}
\begin{Highlighting}[]
\NormalTok{\#| include: false}
\NormalTok{movies |\textgreater{}}
\NormalTok{  summarize(}
\NormalTok{    r\_score\_runtime = cor(runtime, score),}
\NormalTok{    r\_score\_year = cor(year, score)}
\NormalTok{  )}
\end{Highlighting}
\end{Shaded}

\begin{Shaded}
\begin{Highlighting}[]
\NormalTok{\#| eval: false}
\NormalTok{\#| echo: true}
\NormalTok{movies |\textgreater{}}
\NormalTok{  \_\_blank\_1\_\_(}
\NormalTok{    r\_score\_runtime = cor(runtime, score),}
\NormalTok{    r\_score\_year = cor(year, score)}
\NormalTok{  )}
\end{Highlighting}
\end{Shaded}

\begin{verbatim}
# A tibble: 1 × 2
  r_score_runtime r_score_year
            <dbl>        <dbl>
1           0.434. __blank_2__       
\end{verbatim}

\subsubsection{Question 10}\label{question-10}

Which of the following goes in \texttt{\_\_blank\_1\_\_}?

\begin{enumerate}
\def\labelenumi{\alph{enumi}.}
\item
  \texttt{summarize}
\item
  \texttt{mutate}
\item
  \texttt{group\_by}
\item
  \texttt{arrange}
\item
  \texttt{filter}
\end{enumerate}

\subsubsection{Question 11}\label{question-11}

What can we say about the value that goes in \texttt{\_\_blank\_2\_\_}?

\begin{enumerate}
\def\labelenumi{\alph{enumi}.}
\item
  \texttt{NA}
\item
  A value between 0 and 0.434.
\item
  A value between 0.434 and 1.
\item
  A value between 0 and -0.434.
\item
  A value between -1 and -0.434.
\end{enumerate}

\newpage{}

\subsection{Score vs.~runtime and
rating}\label{score-vs.-runtime-and-rating}

In this part, we fit a model predicting \texttt{score} from
\texttt{runtime} and \texttt{rating} (categorized as G, PG, PG-13, R,
NC-17, and Not Rated), and name it \texttt{score\_runtime\_rating\_fit}.

\begin{Shaded}
\begin{Highlighting}[]
\NormalTok{score\_runtime\_rating\_fit \textless{}{-} linear\_reg() |\textgreater{}}
\NormalTok{  fit(score \textasciitilde{} runtime + rating, data = movies)}
\end{Highlighting}
\end{Shaded}

The model output for \texttt{score\_runtime\_rating\_fit} is shown in
\textbf{?@tbl-score-runtime-rating-tidy}. Answer all questions in this
part based on \textbf{?@tbl-score-runtime-rating-tidy}.

\begin{Shaded}
\begin{Highlighting}[]
\NormalTok{\#| label: tbl{-}score{-}runtime{-}rating{-}tidy}
\NormalTok{\#| tbl{-}cap: Regression output for \textasciigrave{}score\_runtime\_rating\_fit\textasciigrave{}.}
\NormalTok{\#| echo: false}
\NormalTok{tidy(score\_runtime\_rating\_fit) |\textgreater{}}
\NormalTok{  knitr::kable(digits = 3)}
\end{Highlighting}
\end{Shaded}

\vspace{1cm}

\subsubsection{Question 12}\label{question-12}

Which of the following is \ul{\textbf{TRUE}} about the intercept of
\texttt{score\_runtime\_rating\_fit}? \textbf{Select all that are true.}

\begin{enumerate}
\def\labelenumi{\alph{enumi}.}
\item
  Keeping runtime constant, G-rated movies are predicted to score, on
  average, 4.525 points.
\item
  Keeping runtime constant, movies without a rating are predicted to
  score, on average, 4.525 points.
\item
  Movies without a rating that are 0 minutes in length are predicted to
  score, on average, 4.525 points.
\item
  All else held constant, movies that are 0 minutes in length are
  predicted to score, on average, 4.525 points.
\item
  G-rated movies that are 0 minutes in length are predicted to score, on
  average, 4.525 points.
\end{enumerate}

\newpage{}

\subsubsection{Question 13}\label{question-13}

Which of the following is the \ul{\textbf{best}} interpretation of the
slope of \texttt{runtime} in \texttt{score\_runtime\_rating\_fit}?

\begin{enumerate}
\def\labelenumi{\alph{enumi}.}
\item
  All else held constant, as runtime increases by 1 minute, the score of
  the movie increases by 0.021 points.
\item
  For G-rated movies, all else held constant, as runtime increases by 1
  minute, the score of the movie increases by 0.021 points.
\item
  All else held constant, for each additional minute of runtime, movie
  scores will be higher by 0.021 points on average.
\item
  G-rated movies that are 0 minutes in length are predicted to score
  0.021 points on average.
\item
  For each higher level of rating, the movie scores go up by 0.021
  points on average.
\end{enumerate}

\subsubsection{Question 14}\label{question-14}

Fill in the blank:

\begin{quote}
R-squared for \texttt{score\_runtime\_rating\_fit} (the model predicting
\texttt{score} from \texttt{runtime} and \texttt{rating})
\_\_\_\_\_\_\_\_\_ the R-squared the model \texttt{score\_runtime\_fit}
(for predicting \texttt{score} from \texttt{runtime} alone).
\end{quote}

\begin{enumerate}
\def\labelenumi{\alph{enumi}.}
\item
  is less than
\item
  is equal to
\item
  is greater than
\item
  cannot be compared (based on the information provided) to
\item
  is both greater than and less than
\end{enumerate}

\newpage{}

\subsubsection{Question 15}\label{question-15}

The model \texttt{score\_runtime\_rating\_fit} (the model predicting
\texttt{score} from \texttt{runtime} and \texttt{rating}) can be
visualized as parallel lines for each level of \texttt{rating}. Which of
the following is the equation of the line for R-rated movies?

\begin{enumerate}
\def\labelenumi{\alph{enumi}.}
\item
  \(\widehat{score} = (4.525 - 0.257) + 0.021 \times runtime\)
\item
  \(score = (4.525 - 0.257) + 0.021 \times runtime\)
\item
  \(\widehat{score} = 4.525 + (0.021 - 0.257) \times runtime\)
\item
  \(score = 4.525 + (0.021 - 0.257) \times runtime\)
\item
  \(\widehat{score} = (4.525 + 0.021) - 0.257 \times runtime\)
\end{enumerate}

\section{Miscellaneous}\label{miscellaneous}

\subsection{Question 16}\label{question-16}

Which of the following is the definition of a regression model? Select
all that apply.

a. \(\hat{y} = b_0 + b_1 X_1\)

b. \(y = \beta_0 + \beta_1 X_1\)

c. \(\hat{y} = \beta_0 + \beta_1 X_1 + \epsilon\)

d. \(y = \beta_0 + \beta_1 X_1 + \epsilon\)

\subsection{Question 17}\label{question-17}

Define the term ``parsimonious model''.

\(\vspace{2cm}\)

\newpage{}

\section{Building a spam filter}\label{building-a-spam-filter}

The data come from incoming emails in David Diez's (one of the authors
of OpenIntro textbooks) Gmail account for the first three months of
2012. All personally identifiable information has been removed. The
dataset is called \texttt{email} and it's in the \textbf{openintro}
package.

The outcome variable is \texttt{spam}, which takes the value \texttt{1}
if the email is spam, \texttt{0} otherwise.

\subsection{Question 18}\label{question-18}

\begin{enumerate}
\def\labelenumi{\alph{enumi}.}
\item
  What type of variable is \texttt{spam}? What percent of the emails are
  spam?
\item
  What type of variable is \texttt{dollar} - number of times a dollar
  sign or the word ``dollar'' appeared in the email? Visualize and
  describe its distribution, supporting your description with the
  appropriate summary statistics.
\item
  Fit a logistic regression model predicting \texttt{spam} from
  \texttt{dollar}. Then, display the tidy output of the model.
\item
  Using this model and the \texttt{predict()} function, predict the
  probability the email is spam if it contains 5 dollar signs. Based on
  this probability, how does the model classify this email?
\end{enumerate}

\begin{tcolorbox}[enhanced jigsaw, breakable, coltitle=black, opacityback=0, colbacktitle=quarto-callout-note-color!10!white, colback=white, titlerule=0mm, leftrule=.75mm, left=2mm, opacitybacktitle=0.6, toprule=.15mm, toptitle=1mm, colframe=quarto-callout-note-color-frame, bottomtitle=1mm, bottomrule=.15mm, title=\textcolor{quarto-callout-note-color}{\faInfo}\hspace{0.5em}{Note}, arc=.35mm, rightrule=.15mm]

To obtain the predicted probability, you can set the \texttt{type}
argument in \texttt{predict()} to \texttt{"prob"}.

\end{tcolorbox}

\newpage{}

\subsection{Question 19}\label{question-19}

\begin{enumerate}
\def\labelenumi{\alph{enumi}.}
\item
  Fit another logistic regression model predicting \texttt{spam} from
  \texttt{dollar}, \texttt{winner} (indicating whether ``winner''
  appeared in the email), and \texttt{urgent\_subj} (whether the word
  ``urgent'' is in the subject of the email). Then, display the tidy
  output of the model.
\item
  Using this model and the \texttt{augment()} function, classify each
  email in the \texttt{email} dataset as spam or not spam. Store the
  resulting data frame with an appropriate name and display the data
  frame as well.
\item
  Using your data frame from the previous part, determine, in a single
  pipeline, and using \texttt{count()}, the numbers of emails:

  \begin{itemize}
  \tightlist
  \item
    that are labelled as spam that are actually spam
  \item
    that are not labelled as spam that are actually spam
  \item
    that are labelled as spam that are actually not spam
  \item
    that are not labelled as spam that are actually not spam
  \end{itemize}

  Store the resulting data frame with an appropriate name and display
  the data frame as well.
\item
  In a single pipeline, and using \texttt{mutate()}, calculate the false
  positive and false negative rates. In addition to these numbers
  showing in your R output, you must write a sentence that explicitly
  states and identified the two rates.
\end{enumerate}

\newpage{}

\subsection{Question 20}\label{question-20}

\begin{enumerate}
\def\labelenumi{\alph{enumi}.}
\item
  Fit another logistic regression model predicting \texttt{spam} from
  \texttt{dollar} and another variable you think would be a good
  predictor. Provide a 1-sentence justification for why you chose this
  variable. Display the tidy output of the model.
\item
  Using this model and the \texttt{augment()} function, classify each
  email in the \texttt{email} dataset as spam or not spam. Store the
  resulting data frame with an appropriate name and display the data
  frame as well.
\item
  Using your data frame from the previous part, determine, in a single
  pipeline, and using \texttt{count()}, the numbers of emails:

  \begin{itemize}
  \tightlist
  \item
    that are labelled as spam that are actually spam
  \item
    that are not labelled as spam that are actually spam
  \item
    that are labelled as spam that are actually not spam
  \item
    that are not labelled as spam that are actually not spam
  \end{itemize}

  Store the resulting data frame with an appropriate name and display
  the data frame as well.
\item
  In a single pipeline, and using \texttt{mutate()}, calculate the false
  positive and false negative rates. In addition to these numbers
  showing in your R output, you must write a sentence that explicitly
  states and identified the two rates.
\item
  Based on the false positive and false negatives rates of this model,
  comment, in 1-2 sentences, on which model (one from Question 19 or
  Question 20) is preferable and why.
\end{enumerate}




\end{document}
